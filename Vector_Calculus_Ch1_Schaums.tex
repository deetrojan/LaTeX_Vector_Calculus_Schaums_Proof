\documentclass{article}

\usepackage{amsmath}

\title{\textit{Hints} to Solutions and Proofs of Some Supplementary Problems of Chapter 1 : \textbf{Vectors and Scalars} of \textit{Schaum's Outlines of Vector Analysis, Second Edition}}
\author{Deepak Thanvi}
\date{17 March 2021}

\begin{document}

\maketitle

\section{Vectors and Scalars}

	\subsection{Question 1.34}
	
	The first important thing to do in mathematics is to draw figures and imagine it.\\
	Draw a figure of ABCDEF Regular Hexagon with forces represented as vectors \textbf{AB},\textbf{AC},\textbf{AD},\textbf{AE} and \textbf{AF}.\\
	
	Now, observe by looking at the figure that :\\
	
	\textbf{AB}=\textbf{ED} \\
	
	\textbf{AF}=\textbf{CD} \\
	
	\textbf{BC}=\textbf{FE} \\
	 
	Let the Resultant force between \textbf{AB} and \textbf{AC} is \textbf{$R_{1}$}, \textbf{AF} and 				\textbf{AE} is \textbf{$R_{2}$}, \textbf{$R_{2}$} and \textbf{AD} is \textbf{$R_{3}$}, \textbf{$R_{3}$} 		and \textbf{$R_{1}$} is \textbf{$R_{4}$} respectively.\\
	
	Now,\\
	\textbf{$R_{4}$}=\textbf{$R_{1}$}+\textbf{$R_{3}$} \\
	
	Using above all obervations, Simplifying \textbf{$R_{4}$} \\
	
	\textbf{$R_{4}$}=\textbf{AB}+\textbf{AC}+\textbf{$R_{2}$}+\textbf{AD}\\
	
	\textbf{$R_{4}$}=\textbf{AB}+\textbf{AC}+\textbf{AF}+\textbf{AE}+\textbf{AD} \\
	
	Also, \\
	\textbf{AC}=\textbf{AB}+\textbf{BC}\\
	
	\textbf{AE}=\textbf{AF}+\textbf{FE}\\
	
    Considering all the observations and using it in previous expression of \textbf{$R_{4}$}  \\
    
	\textbf{$R_{4}$}=\textbf{AB}+\textbf{AC}+\textbf{AF}+\textbf{AE}+\textbf{AD} \\
	
	\textbf{$R_{4}$}=\textbf{AB}+\textbf{AB}+\textbf{BC}+\textbf{AF}+\textbf{AF}+\textbf{FE}+\textbf{AD}\\
	
	Since, \textbf{AB}+\textbf{BC}+\textbf{CD}=\textbf{AD},\\
	and \\
	
	\textbf{AB}=\textbf{ED}\\
	
	\textbf{AF}=\textbf{CD}\\
	
	\textbf{BC}=\textbf{FE}\\
	
	\textbf{$R_{4}$}=3\textbf{AD} \\
	
	Hence, the resultant of the forces is 3\textbf{AD}. 
	
	
	
	\subsection{Question 1.39}	
	
	Given :\\
	
	\textbf{A}=(x+4y)\textbf{a}+(2x+y+1)\textbf{b}\\

	\textbf{B}=(y-2x+2)\textbf{a}+(2x-3y-1)\textbf{b}\\
	
	Now, 3\textbf{A}=2\textbf{B} \\
	
	3\textbf{A}=2\textbf{B}\\
	
	3\textbf{A}-2\textbf{B}=0\\
	
$\Rightarrow$ (3x+12y)\textbf{a}+(6x+3y+3)\textbf{b}-(2y-4x+4)\textbf{a}+(4x-6y-2)\textbf{b}=0\\

$\Rightarrow$ (7x+10y-4)\textbf{a}+(2x+9y+5)\textbf{b}=0\\

    Since, \textbf{a} and \textbf{b} are \textbf{Non-Collinear},\\
    
    7x+10y=4,\\
    
    2x+9y=-5\\
    
    By solving above equations, we get x=2 and y=-1.
    
    \subsection{Question 1.43}
    
    It is given that \textbf{a},\textbf{b} and \textbf{c} are non-coplanar vectors.\\
    
    Also,\\
    
    \textbf{$r_{1}$} = 2\textbf{a} - 3\textbf{b}+\textbf{c} ,\\
    
    \textbf{$r_{2}$} = 3\textbf{a} - 5\textbf{b} + 2\textbf{c} ,\\
    
    \textbf{$r_{3}$} = 4\textbf{a} - 5\textbf{b} + \textbf{c} ,\\
    
    We need to check whether \textbf{$r_{1}$},\textbf{$r_{2}$} and \textbf{$r_{3}$} are linearly independent or linearly dependent.\\
    
    Let L = x\textbf{$r_{1}$} + y\textbf{$r_{2}$} + c\textbf{$r_{3}$} be a Linear Combination of \textbf{$r_{1}$},\textbf{$r_{2}$} and \textbf{$r_{3}$}, where x,y and z are any scalars.\\
    
    L = x\textbf{$r_{1}$} + y\textbf{$r_{2}$} + c\textbf{$r_{3}$}\\
    
    L = x(2a - 3b + 2c) + y(3a - 5b + 2c) + z(4a - 5b + c) \\
    
    L = (2x + 3y + 4z)a + (-3x - 5y - 5z)b + (x + 2y + z)c \\
    
    Since, \textbf{a},\textbf{b} and \textbf{c} are non-coplanar vectors, \\
    
    L = 0 and \\
    
    2x + 3y + 4z = 0\\
    
   -3x - 5y - 5z = 0\\
    
    x + 2y + z = 0\\
    
    By solving above equations, we get y = 2z and x = -5z.\\
    
    Taking z = 1, yields y = 2 and x = -5.\\
    
    Hence,
    L = 0 \\
$\Rightarrow$ x\textbf{$r_{1}$} + y\textbf{$r_{2}$} + c\textbf{$r_{3}$} = 0\\

$\Rightarrow$ 5\textbf{$r_{1}$} = 2\textbf{$r_{2}$} + \textbf{$r_{3}$} \\

    Therefore, \textbf{$r_{1}$},\textbf{$r_{2}$} and \textbf{$r_{3}$} are linearly dependent.
    
    
    \subsection{Question 1.44}
    
    Part (a) \\
    
    It is given that if O is any point within triangle ABC and P,Q, and R are midpoints of the sides AB,BC and CA respectively.\\
    
    We need to prove that :\\
    \textbf{OA}+\textbf{OB}+\textbf{OC}=\textbf{OP}+\textbf{OQ}+\textbf{OR}\\
    
    First of all, we need to draw a triangle ABC with sides representing vectors.\\
    
    By looking at the figure, we find : \\
    
    \textbf{AP}=\textbf{PB}\\
    
    \textbf{BQ}=\textbf{QC}\\
    
    \textbf{CR}=\textbf{RA}\\
    
    
\[
    \textbf{OA} =
    \begin{cases}
    		1. & \text{\textbf{OR}+\textbf{RA}}\\
    		2. & \text{\textbf{OP}-\textbf{AP}}
    \end{cases}
\]

\[
    \textbf{OB} =
    \begin{cases}
    		3. & \text{\textbf{OP}+\textbf{PB}}\\
    		4. & \text{\textbf{OQ}-\textbf{BQ}}
    \end{cases}
\]

\[
    \textbf{OC} =
    \begin{cases}
    		5. & \text{\textbf{OQ}+\textbf{QC}}\\
    		6. & \text{\textbf{OR}-\textbf{CR}}
    \end{cases}
\]

	Adding all above 6 equations, \\
	
	2(\textbf{OA}+\textbf{OB}+\textbf{OC})= (\textbf{OR}+\textbf{RA}+\textbf{OP}+\textbf{PB}+\textbf{OQ}+\textbf{QC}) + (\textbf{OP}-\textbf{AP}+\textbf{OQ}-\textbf{BR}+\textbf{OR}-\textbf{CR})\\
	
$\Rightarrow$ 	2(\textbf{OP}+\textbf{OQ}+\textbf{OR}) + (\textbf{RA}-\textbf{AP}+\textbf{PB}-\textbf{CR}+\textbf{QC}-\textbf{BR})\\

$\Rightarrow$	\textbf{OA}+\textbf{OB}+\textbf{OC}=\textbf{OP}+\textbf{OQ}+\textbf{OR}\\

	Hence Proved.\\
	
	Part (b) \\
	
	Yes. You can do on your own.\\
	\textbf{Hint}: Consider Point O outside the triangle ABC and follow Part (a).
	
	
	\subsection{Question 1.46}
	
	Consider a triangle ABC with vertices \textbf{AC},\textbf{CB} and \textbf{BC} and with mid point P of \textbf{BA} joining Q, the mid point of \textbf{CB}.\\
	
	We need to prove that \textbf{PQ}$\parallel$\textbf{AC} and PQ=$\frac{1}{2}$AC\\
	
	First draw the figure with given conditions. By looking at the figure : \\
	
	\textbf{BP} = \textbf{PA} = $\frac{1}{2}$\textbf{BA}\\
	
	\textbf{CQ} = \textbf{QB} = $\frac{1}{2}$\textbf{CB}\\
	
	Also, \\
	
	\textbf{PQ} = \textbf{QB} + \textbf{BP}\\
	
$\Rightarrow$ $\frac{1}{2}$\textbf{CB} + $\frac{1}{2}$\textbf{BA}\\

$\Rightarrow$ $\frac{1}{2}$(\textbf{CB} + \textbf{BA})\\

$\Rightarrow$ PQ = $\frac{1}{2}$AC\\

$\Rightarrow$ \textbf{PQ}$\parallel$\textbf{AC} and PQ=$\frac{1}{2}$AC\\

	Hence Proved.
	
		
    

	
	

\end{document}